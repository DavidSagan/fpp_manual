% !TEX TS-program = PDFlatex
% !TEX encoding = UTF-8 Unicode

% Example of the Memoir class, an alternative to the default LaTeX classes such as article and book, with many added features built into the class itself.

%\documentclass[12pt,a4paper]{memoir} % for a long document
%\documentclass[french,12pt,a4paper,article]{memoir} % for a short document
\documentclass[english,12pt,article]{article} % for a short document
%\documentclass[french,12pt,a4paper]{article} % for a short document

\usepackage{babel}
\usepackage[utf8]{inputenc} % set input encoding to utf8
 \usepackage{amsmath, amsthm, amssymb}
\input{amssym.def}
\input{amssym.tex}

% Don't forget to read the Memoir manual: memman.PDF
\usepackage{color}
\usepackage{pgfplotstable}
\usepackage{pgfplots}
\usetikzlibrary{plotmarks}
\usepackage[colorlinks=true,linkcolor=blue]{hyperref} 
\usepackage{graphicx}
%%% Examples of Memoir customization
%%% enable, disable or adjust these as desired

%%% PAGE DIMENSIONS
% Set up the paper to be as close as possible to both A4 & letter:
%\settrimmedsize{11in}{210mm}{*} % letter = 11in tall; a4 = 210mm wide
%\setlength{\trimtop}{0pt}
%\setlength{\trimedge}{\stockwidth}
%\addtolength{\trimedge}{-\paperwidth}
%\settypeblocksize{*}{\lxvchars}{1.618} % we want to the text block to have golden proportionals
%\setulmargins{50pt}{*}{*} % 50pt upper margins
%\setlrmargins{*}{*}{1.618} % golden ratio again for left/right margins
%\setheaderspaces{*}{*}{1.618}
%\checkandfixthelayout 
% This is from memman.PDF

\newcommand{\latex}{\LaTeX\xspace}
\newcommand{\tex}{\TeX\xspace}
 
 \def\th#1{theorem~\ref{#1}}
\def\Th#1{Theorem~\ref{#1}}
\def\eq#1{\rm Eq.~(\ref{#1})}
\def\Eq#1{\rm Eq.~(\ref{#1})}
\def\eqs#1{\rm Eqs.~(\ref{#1})}
\def\eqe#1{(\ref{#1})}
\def\sec#1{\rm Sec.~(\ref{#1})}
\def\secs#1{\rm Secs.~(\ref{#1})}
\def\sece#1{(\ref{#1})}
\def\Fig#1{\rm Figure {\rm \ref{#1}}}
\def\fig#1{\rm Figure {\rm \ref{#1}}}
\def\fige#1{ {\rm \ref{#1}}}
\def\figs#1{\rm Figures {\rm \ref{#1}}}

\definecolor{darkgreen}{rgb}{0,0.8,0.2}
\definecolor{mauve}{rgb}{0.88,0.69,1.0}
\definecolor{morve}{rgb}{0.58,0.7,0.39}
\definecolor{darkred}{rgb}{0.58,0.0,0.0}
%\definecolor{ered}{rgb}{1.0,0.0,0.0}
\definecolor{ered}{rgb}{0.0,0.0,0.0}   % ered devient noir
\definecolor{pion}{rgb}{0.01.0,1.0}
%\frenchspacing
\usepackage{pdfpages}
 

\title{ \protect \large Manual for FPP:   The Fully Polymorphic Package}    
\newcommand{\subtitle}{Using examples from  PTC and BMAD }
\author{Étienne Forest \\ Tsukuba, Japon }
%\date{12 avril 2018} 
%\date{21 juillet 2018} % Delete this line to display the current date


 \bibliographystyle{prsty}
 

 
\begin{document}
 
\maketitle

\newpage 
{\footnotesize
\tableofcontents % the asterisk means that the contents itself isn't put into the ToC
}
\newpage 

\section{Distinguishing PTC from FPP}\label{sec:fppptc}

 \subsection{Tracking and analysis }\label{sub:ta}
 
 In E.\,Forest's book \cite{thebook}, it is alleged that accelerator theory and simulation can be cleanly separated into two parts:
 
 \begin{enumerate}
 \item\label{i1} A tracking code which simply tracks rays, for example  $(q,p)$,  if the code has only one degree of freedom (1-d-f). 
 \item\label{i2}  An analysis part which computes lattice functions, tunes, tune shifts, etc\dots via Taylor series map produced by the tracking code. 
 \end{enumerate}
 
 If a code simply tracks rays, i.e., two real variables  $(q,p)$,  it does not seem feasible that Taylor series maps will be magically produced. Therefore it may not seem possible that items \ref{i1} and \ref{i2} can co-exist within the same programming environment.
 
 
 However here enters the magic of Truncated Power Series Algebra introduced in our field by Martin Berz\cite{tpsa}. With little effort, a code which pushes real numbers can be converted into a code which tracks Taylor series. For example, a code denoted by $T$ which tracks the closed orbit of a ring, say  $(q_0,p_0)$, can be coerced to track the linear matrix approximation around this orbit:
 
%
%]|Expr|[#b @`b___})b!3# b'4" Chicago^: ;bP8&c0!*-="!Helvetica|:! |
%|""Monaco^:"if  "#*|:#T: ,H:! $^:#q^0_:! -<cr $^:#p^0_:! : ,I:# |
%|:!,F,]:# : ,H:! $^:#q^0_:! -<cr $^:#p^0_:! : ,I;bP;/":!;bP8 :#T|
%|: ,H:! $^:#q^0_,Kd$^z^1_:! -<cr $^:#p^0_,Kd$^z^2_:! : ,I:!,F,]|
%|: ,H:! $^:#q^0_:! -<cr $^:#p^0_:! : ,I:! ,K <c%"C^<cY A^<c!$1|
%|^["" ^$^m(!11}_)!# b'4$^m(!12}_}& b!( b"0 b#8 b$@ b%H b&P!WW}|
%|^$^m(!21}_^$^m(!22}_}}}^M_}: ,H:! :#d$^z^1_:! -<cr :#d$^z^2_:!|
%| : ,I :# : ;8/<:#;bP8eqtpsa: ;8/=;bP8-;|
%|}& b!( b"0 b#8 b$@ b%H b&P!WW}]|[
\begin{align} {\rm i}{\rm f}~~T\left({\begin{array}{c} {q}_{0} \cr {p}_{0} \end{array}}\right)\ &=\ \left({\begin{array}{c} {q}_{0} \cr {p}_{0} \end{array}}\right)\nonumber \\
 T\left({\begin{array}{c} {q}_{0}+d{z}_{1} \cr {p}_{0}+d{z}_{2} \end{array}}\right)&=\left({\begin{array}{c} {q}_{0} \cr {p}_{0} \end{array}}\right) + \underbrace{\left({\begin{array}{cc}{m}_{11}&{m}_{12}\\
{m}_{21}&{m}_{22}\end{array}}\right)}\limits_{M}^{}\left({\begin{array}{c} d{z}_{1} \cr d{z}_{2} \end{array}}\right) \ \label{eqtpsa}\end{align}
%
It is well-known that knowing the matrix $M$ allows a user to compute the lattice functions and the tune:
%
%]|Expr|[#b @`b___})+# b'4" Chicago^: ;bP8&c0!*-="!Helvetica|:! |
%|""Times|:"M:!,F,]<c!$1^["" ^$^:"m(!11}_)!# b'4$^m(!12}_|
%|}& b!( b"0 b#8 b$@ b%H b&P!WW}^$^m(!21}_^$^m(!22}_}},]<c!$1^["" |
%|(*:!-<cos<c!$1^"#Symbol^:#&c0  m}"$*|:$&c0!* ,K :#&c0  a:!&c0!*-<|
%|sin<c!$1^:#&c0  m}}($b:!&c0!*-<sin<c!$1^:#&c0  m}}(%:$&c0!*,M|
%|:#&c0  g:!&c0!*-<sin<c!$1^:#&c0  m}}(*:!&c0!*-<cos<c!$1^:#&c0  m}|
%|:$&c0!* ,M :#&c0  a:!&c0!*-<sin<c!$1^:#&c0  m}}}}: ;8&c0!*/<:$;bP8eqcs|
%|: ;8/=;bP8-;}& b!( b"0 b#8 b$@ b%H b&P!WW}]|[
\begin{align} M&=\left({\begin{array}{cc}{m}_{11}&{m}_{12}\\
{m}_{21}&{m}_{22}\end{array}}\right)=
\left({\begin{array}{cc}\cos\left({\mu }\right)\ +\ \alpha \sin\left({\mu }\right)&\beta \sin\left({\mu }\right)\\
-\gamma \sin\left({\mu }\right)&\cos\left({\mu }\right)\ -\ \alpha \sin\left({\mu }\right)\end{array}}\right)\label{eqcs}\end{align}

 The code $T$, which we can view  as a function,  must be instructed to take derivatives with respect to the phase space. If one evaluates 
%
%
%]|Expr|[#b @`b___})1# b'4" Chicago^: ;bP8&c0!*-="!Helvetica|:! |
%|""*|:"T: ,H:! $^:"q^0_,Kd$^z^1_:! -<cr $^:"p^0_: ,I:! : .O-;|
%|}& b!( b"0 b#8 b$@ b%H b&P!WW}]|[
\begin{align} T\left({\begin{array}{c} {q}_{0}+d{z}_{1} \cr {p}_{0}\end{array}}\right) \nonumber 
\end{align}
%
then the output will be a Taylor series but the output {\bf cannot}  be considered an approximate rendition of the map/code $T$. It is simply a Taylor expansion of $T$ with respect to the first phase space coordinate.  

Additionally one can expand $T$ with respect to system parameters: multipole strengths, lengths, etc\ldots 
The resulting Taylor expansion will not be a {\it bona fide} approximate map of $T$ unless the vector of infinitesimals $(dz_1,dz_2)$ is added to the initial $(q_0,p_0)$ by the magic of TPSA .

In conclusion, the code PTC tracks rays and also produces Taylor series. The library FPP can analyse the output if and only if the approximate map is a {\it bona fide} approximate map of $T$.  As we will see, an enormous amount of things can be computed: lattice functions, phase advance for orbital and spin dynamics as described by Forest 
in  \cite{thebook2}.

It is therefore important to distinguish a Taylor expansion of the ray and the {\it bone fide} approximate maps.  We will see this in \sec{sub:map}. But first let us introduce a Taylor type and a polymorphic type

 \subsection{Taylor series and Polymorphs}\label{sub:pol}

As is hinted in the previous section, the programmer does not {\it a priori} if the ray will be a Taylor series or simply real numbers. To deal with this issue, we created a polymorphic type called real\_8. Here is its definition found in the file h\_definition.f90 of FPP(PTC). We also include the definition of Taylor.

{\footnotesize
\begin{verbatim}
  TYPE TAYLOR
     INTEGER I !  integer I is a pointer in old da-package of Berz
  END TYPE TAYLOR


  TYPE REAL_8
     TYPE (TAYLOR) T      !  USED IF TAYLOR
     REAL(DP) R           !    USED IF REAL
     INTEGER KIND  !  0,1,2,3 (1=REAL,2=TAYLOR,3=TAYLOR KNOB, 0=SPECIAL)
     INTEGER I   !   USED FOR KNOBS AND SPECIAL KIND=0
     REAL(DP) S   !   SCALING FOR KNOBS AND SPECIAL KIND=0
     LOGICAL(LP) :: ALLOC  1 IF TAYLOR IS ALLOCATED IN DA-PACKAGE
  END TYPE REAL_8
\end{verbatim}
}

One notes that Taylor is simply an integer. This integer points to the $I^{th}$ Taylor series in the LBNL version of Berz' package. In that package all the Taylor series use the same amount of memory. For example,  a third order Taylor series in two variables would contain $(3+2)!/3!/2!$ terms, namely 10 terms:
%
%]|Expr|[#b @`b___})b H# b'4" Chicago^: ;bP8&c0!*-="!Helvetica|:! |
%|""*|:"t:!,F,]$^:"t(!00}_,K$^t(!10}_$^z^1_,K$^t(!01}_$^z^2_,K$|
%|^t(!20}_$^z^1^2,K$^t(!11}_$^z^1_$^z^2_,K$^t(!02}_$^z^2^2: ;bP;/"|
%|:!;bP8 ,F:",K$^t(!30}_$^z^1^3,K$^t(!21}_$^z^1^2$^z^2_,K$^t(!12}|
%|_$^z^1_$^z^2^2,K$^t(!03}_$^z^2^3:! : -;|
%|}& b!( b"0 b#8 b$@ b%H b&P!WW}]|[
\begin{align} t&={t}_{00}+{t}_{10}{z}_{1}+{t}_{01}{z}_{2}+{t}_{20}{z}_{1}^{2}+{t}_{11}{z}_{1}{z}_{2}+{t}_{02}{z}_{2}^{2}\nonumber \\
 &+{t}_{30}{z}_{1}^{3}+{t}_{21}{z}_{1}^{2}{z}_{2}+{t}_{12}{z}_{1}{z}_{2}^{2}+{t}_{03}{z}_{2}^{3} \end{align}
%
In any code the size of PTC it would be totally infeasible memory-wise to make every variable of the code of type Taylor. Instead we invented the type real\_8. The type real\_8 contains a type taylor which is not ``activated'' if the polymorph is real but suddenly becomes Taylor if the said polymorph must be a Taylor series. This is illustrated in the program z-tpsa0.f90.  


{\footnotesize
\begin{verbatim}
program example0
use madx_ptc_module
use pointer_lattice
implicit none
type(real_8) x_8
real(dp) x
longprint=.false.
call ptc_ini_no_append
call append_empty_layout(m_u)
call set_pointers; use_quaternion=.true.;

call init(only_2d0,3,0) ! TPSA set no=3 and nv=2

x=0.1d0

call alloc(x_8)
 
x_8=x  ! insert a real into a polymorph
 
write(6,*) " Must be real "
call print(x_8)
 
x_8=x_8+dz_8(1)
 
write(6,*) " Must be Taylor : 0.1+dx "
call print(x_8)

x_8=x_8**4
 
write(6,*) " Must (0.1+dx)**4 to third order "
 
call print(x_8)
end program example0
\end{verbatim}
}


The result of this program is:

{\scriptsize
\begin{verbatim}
  Must be real
  0.100000000000000
  Must be Taylor : 0.1+dx

 Properties, NO =    3, NV =    2, INA =   21
 *********************************************

   0  0.1000000000000000       0  0
   1   1.000000000000000       1  0

  Must (0.1+dx)**4 to third order

 Properties, NO =    3, NV =    2, INA =   21
 *********************************************

   0  0.1000000000000000E-03   0  0
   1  0.4000000000000001E-02   1  0
   2  0.6000000000000001E-01   2  0
   3  0.4000000000000000       3  0
\end{verbatim}
}
The last output is simply:
%
%]|Expr|[#b @`b___}):# b'4" Chicago^: ;bP8&c0!*-="!Helvetica|:! |
%|""*|:"x-?8:!,F,]$(!:"10}_(",M3},K4"#Symbol^:#&c0  .T$(!:"&c0!*10}|
%|_(",M3}$^z^1_,K6:#&c0  .T$(!:"&c0!*10}_(",M2}$^z^1^2,K4:#&c0  .T|
%|$(!:"&c0!*10}_(",M1}$^z^1^3,K'b M:! : -;|
%|}& b!( b"0 b#8 b$@ b%H b&P!WW}]|[
\begin{align} x\_8&={10}^{-3}+4\times {10}^{-3}{z}_{1}+6\times {10}^{-2}{z}_{1}^{2}+4\times {10}^{-1}{z}_{1}^{3}+\cdots \end{align}
%
The code behaved as expected: once the infinitesimal  $dz_1$ (stored in dz\_8(1)) is added to x\_8, the entire expression must become a Taylor map.
In our example, the assignment x\_8=x\_8+dz\_8(1), forces the variable x\_8\%kind to be  set to 2 and  x\_8\%alloc  to true. The polymorph is now the Taylor series  x\_8\%t.


We now look at the probe  and the probe\_8 of the code PTC which contains all the trackable objects.: phase space and  spin primarily.

\subsection{ map}\label{sub:map}
\bibliography{nlinear}


\end{document}



